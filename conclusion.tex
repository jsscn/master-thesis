\chapter{Conclusion}

% TODO something here? something about data science in general again?

We implemented an algorithm for mining frequent episodes and confident association rules, according to multiple frequency and confidence measures, ranging from long-established approaches to more novel ones. While the general strategy for mining episodes is consistent across these episode classes and frequency measures, quite a few specialized subalgorithms were needed. Many of the subalgorithms were based on preexisting algorithms, though most needed some modification for our purposes.

% TODO ^ expand introduction to conclusion ^

We saw that each of these interestingness measures had advantages and disadvantages. The fixed-window frequency was often, especially with parallel episodes, since the amount of bookkeeping in a database pass is a lot more limited compared to the other approaches. Furthermore, computing the fixed-window confidence of an association rule is trivial, given the frequency of the episodes from which the rule is composed.

% TODO ^ more advantages, disadvantages ^

There are downsides to our implementation.
- efficiency often worse than closed episode miner
- no closed episodes
- no general episodes
- no association rules from both parallel and serial episodes at once
- output quality suffers due to redundancy (closedness plays into this as well)

recap algorithms; different algorithms for both episode classes

experiments

efficiency: episodes (classes, frequency measures, window widths), association rules

output quality: better with closed and stuff

We saw that the computationally less complex interestingness measures often performed better in terms of output size, which can be advantageous when trying to lower the frequency threshold as much as possible. However, more complex measures certainly have advantages of their own: they often make more careful considerations when deciding the fate of a poor episode.

fixed-window frequency/confidence have theoretical disadvantages but practical advantages
other measures more theoretically advantageous but less practically feasible
