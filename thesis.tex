\documentclass{scrartcl}
\usepackage[utf8]{inputenc}

\usepackage[T1]{fontenc}
\usepackage{lmodern}
\usepackage{amsmath}
\usepackage{amssymb}
\usepackage{amsfonts}
\usepackage{bm}
\usepackage{float}
\usepackage{graphicx}
\usepackage{caption} % to remove colon from figures with no caption
% \usepackage[top=3cm, left=2cm, right=2cm, bottom=3cm]{geometry}

\usepackage[colorlinks=true,urlcolor=blue]{hyperref}
\usepackage{algorithm}
\usepackage[noend]{algorithmic}
\usepackage{listings}
\usepackage{courier}
\usepackage{amsthm}

\usepackage[gen]{eurosym}

\lstset{basicstyle=\footnotesize\ttfamily,breaklines=true}

\renewcommand{\sfdefault}{phv}
\renewcommand{\rmdefault}{bch}

\setlength{\parindent}{0em}
\setlength{\parskip}{0.5em}

\theoremstyle{definition}
\newtheorem{definition}{Definition}

% \renewcommand{\baselinestretch}{1.25}

\begin{document}

\frenchspacing

\title{Evaluation of Algorithms for Sequential Pattern Mining in Long Event Sequences}
\subtitle{}

%***********************************************************************
% AUTHORS INFORMATION AREA
%***********************************************************************
\author{Josse Coen
\vspace{.3cm}\\
%
% Addresses and institutions (remove "1- " in case of a single institution)
University of Antwerp
%
% Remove the next three lines in case of a single institution
% \vspace{.1cm}\\
% 2- School of Second Author - Dept of Second Author \\
% Address of Second Author's school - Country of Second Author's school\\
}
%***********************************************************************
% END OF AUTHORS INFORMATION AREA
%***********************************************************************

\date{1 January 1970}

\maketitle

\section{Introduction}


\section{Problem Statement}

\subsection{Event sequences}

\begin{definition}
A \emph{sequence event} is defined as a pair $ (A, t) $ where $ A \in \Sigma $ is an event type from a given set of event types $ \Sigma $, and $ t $ is a timestamp integer.
\end{definition}

\begin{definition}
An \emph{event sequence} $ \boldsymbol{s} $ is an ordered sequence of events
\begin{align*}
s = \langle (A_1, t_1), (A_2, t_2), \ldots, (A_n, t_n) \rangle
\end{align*}
such that $ t_i \leq t_{i + 1} $ for all $ i = 1, \ldots, n - 1 $.
\end{definition}

If event $ (A, t) $ is in $ s $ for a given sequence $ \boldsymbol{s} $, we say that the event \emph{occurs} in the sequence at timestamp $ t $. Note that multiple events can occur at the same timestamp.

Given a sequence $ \boldsymbol{s} $ and two integers $ i $ and $ j $ we define a \emph{subsequence} $ s[i, j] = s_i, \ldots, s_j $ containing all events occurring between $ i $ and $ j $.

For the sake of simplicity, we use the notation $ s_1 \cdots s_N $ to mean the sequence $ \langle (s_1, 1), \ldots, (s_N, N) \rangle $.

\begin{definition}
An \emph{episode} $ G $ is represented by a directed acyclic graph with labelled nodes, that is, $ G = (V, E, lab) $, where $ V = (v_1, \ldots, v_K) $ is the set of nodes, $ E $ is the set of directed edges, and lab is the function $ lab \colon V \rightarrow \Sigma $, mapping each node $ v_i $ to its event type.
\end{definition}

\begin{definition}
A node $ n $ in an episode graph is a \emph{descendant} of a node $ m $ if there is a path from $ m $ to $ n $. Conversely $ m $ is an \emph{ancestor} of $ n $ in that case.
\end{definition}

\begin{definition}
Given a sequence $ s $ and an episode $ G $ we say that $ s $ \emph{covers} G, or $ G $ \emph{occurs} in $ s $, if there is an injective map $ f $ mapping each node $ v_i $ to a valid index such that the node $ v_i $ in $ G $ and the corresponding sequence element $ s_{f(v_i)} $ have the same label: $ s_{f(v_i)} = lab(v_i) $; and that if there is an edge $ (v_i, v_j) $ in $ G $, then we must have $ f(v_i) < f(v_j) $. In other words, the parents of $ v_j $ must occur in $ s $ before $ v_j $. If the mapping $ f $ is surjective, that is, all events in $ s $ are used, we will say that $ s $ is an \emph{instance} of G.
\end{definition}

\begin{definition}
Given two episodes $ G $ and $ H $, we say that $ G $ is a \emph{subepisode} of $ H $, denoted $ G \subseteq H $, if the graph describing episode $ G $ is a subgraph of the graph describing episode $ H $.
\end{definition}

In this thesis, we'll be limiting ourselves to two subcategories of episodes:
\begin{itemize}
\item \textbf{Parallel episodes.} A parallel episode is an episode for which the set of edges is empty. As such, event types can appear in any order in a sequence.
\item \textbf{Serial episodes.} A serial episode is an episode for which the edges cause the nodes to have a strict order. That is, in any occurrence of an episode in a sequence, the event types appear in the same order.
\end{itemize}

\subsection{Association Rules}
\begin{definition}
Given two episodes $ G $ and $ H $ such that $ G \subset H $, we can express an \emph{association rule} $ G \Rightarrow H $. We call $ G $ the \emph{head} of the rule, and $ H $ the \emph{tail} of the rule.
\end{definition}

\subsection{Algorithms}

Algorithm~\ref{alg:recognize-serial-fixed-win-episodes} for recognizing a collection of episodes $ \mathcal{C} $ in a sequence. By the definition of serial episodes, all nodes must appear in the sequence in a strict order. Serial episodes are therefore recognized using automata, instances of which advance as events are encountered.

Each episode has its own automaton, which consists of $ | \alpha | $ states: each state corresponds to a node in the episode. A state can be represented by the index of the node in the episode it corresponds to. Then an instance of the automaton for $ \alpha $ being in a state $ j $ denotes that the episode has been recognized up to (and including) the $ j $-th node. When in state $ j $ and upon encountering an event of which the type corresponds to the $ (j + 1) $-th node of the episode, the instance will transition to state $ j + 1 $.

When an instance of $ \alpha $ reaches state $ | \alpha | $, the episode has been successfully recognized.

\begin{algorithm}

\caption{Recognizing serial episodes. \\
Input: A collection $ \mathcal{C} $ of serial episodes, an event sequence $ \boldsymbol{s} = (s, T_s, T_e) $, a window width \textit{win}, and a frequency threshold \textit{min\_fr}. \\
Ouptut: The episodes of $ \mathcal{C} $ that are frequent in $ \boldsymbol{s} $ with respect to \textit{win} and \textit{min\_fr}.
}

\begin{algorithmic}[1]

\FORALL{$ \alpha \in \mathcal{C} $}
    \FOR{$ i \leftarrow 1 $ \TO $ | \alpha | $}
        \STATE{$ \alpha \text{.initialized} \leftarrow 0 $}
        \STATE{$ \text{waits}(\alpha[i]) \leftarrow 0 $}
    \ENDFOR
\ENDFOR

\FORALL{$ \alpha \in \mathcal{C} $}
    \STATE{$ \text{waits}(\alpha[1]) \leftarrow \text{waits}(\alpha[1]) \cup \left\{ \left( \alpha, 1 \right) \right\} $}
    \STATE{$ \alpha \text{.freq\_count} \leftarrow 0 $}
\ENDFOR

\FOR{$ t \leftarrow T_s - \text{win} $ \TO $ T_s - 1 $}
    \STATE{$ \text{begins\_at}(t) \leftarrow \emptyset $}
\ENDFOR

\FOR{$ \text{start} \leftarrow T_s - \text{win} + 1 $ \TO $ T_e $}
    \STATE{$ \text{begins\_at}(\text{start} + \text{win} - 1) \leftarrow \emptyset $}
    \STATE{$ \text{transitions} \leftarrow \emptyset $}
    \FORALL{events $ (A, t) $ in $ s $ such that $ t = \text{start} + \text{win} - 1 $}
        \FORALL{$ ( \alpha, j) \in \text{waits}(A) $}
            \IF{$ j = | \alpha | $ and $ \alpha \text{.initialized}[j] = 0 $}
                \STATE{$ \alpha \text{.in\_window} \leftarrow \text{start} $}
            \ENDIF
            \IF{$ j = 1 $}
                \STATE{$ \text{transitions} \leftarrow \text{transitions} \cup \{ ( \alpha, 1, \text{start} + \text{win} - 1 ) \} $}
            \ELSE
                \STATE{$ \text{transitions} \leftarrow \text{transitions} \cup \{ \alpha, j, \alpha \text{.initialized} [j - 1] \} $}
                \STATE{$ \text{begins\_at}( \alpha \text{.initialized}[j - 1] ) \leftarrow \text{begins\_at}( \alpha \text{.initialized}[j - 1] ) \setminus \{ ( \alpha, j - 1 ) \} $}
                \STATE{$ \alpha \text{.initialized} [j - 1] \leftarrow 0 $}
                \STATE{$ \text{waits}(A) \leftarrow \text{waits}(A) \setminus \{ ( \alpha, j ) \} $}
            \ENDIF
        \ENDFOR
    \ENDFOR
    \FORALL{$ ( \alpha, j, t ) \in \text{transitions} $}
        \STATE{$ \alpha \text{.initialized} [j] \leftarrow t $}
        \STATE{$ \text{begins\_at}(t) \leftarrow \text{begins\_at}(t) \cup \{ ( \alpha, j ) \} $}
        \IF{$ j < | \alpha | $}
            \STATE{$ \text{waits}(\alpha [j + 1]) \leftarrow \text{waits}(\alpha [j + 1]) \cup \{ (\alpha, j + 1) \} $}
        \ENDIF
    \ENDFOR
    \FORALL{$ (\alpha, l) \in \text{begins\_at}(\text{start} - 1) $}
        \IF{$ l = | \alpha | $}
            \STATE{$ \alpha \text{.freq\_count} \leftarrow \alpha \text{.freq\_count} - \alpha \text{.in\_window} + \text{start} $}
        \ELSE
            \STATE{$ \text{waits}(\alpha [l + 1]) \leftarrow \text{waits}(\alpha [l + 1]) \setminus \{ ( \alpha, l + 1 ) \} $}
        \ENDIF
        \STATE{$ \alpha \text{.initialized}[l] \leftarrow 0 $}
    \ENDFOR
\ENDFOR
\FORALL{episodes $ \alpha $ in $ \mathcal{C} $}
    \IF{$ \alpha \text{.freq\_count} / T_e - T_s + \text{win} - 1 \geq \text{min\_fr} $}
        \STATE{output $ \alpha $}
    \ENDIF
\ENDFOR

\end{algorithmic}

\label{alg:recognize-serial-fixed-win-episodes}
\end{algorithm}


\section{Implementation}


\section{Experiments}



\section{Conclusion}

\end{document}
