\chapter{Experiments}
\label{sec:experiments}

In this section, we perform experiments to answer the research question we posed in the introduction. Concisely put, we want to know whether we can find patterns in long sequences in a reasonable time, and whether there are interesting patterns to be found.

After choosing a number of datasets and showing some of their basic characteristics, we assess the performance of the implementation by running it on datasets of different kinds, and varying the parameter.

First, we'll briefly discuss how we ensured the correctness of our implementation.

\section{Correctness}

A big point of comparison while implementing our algorithms was a closed episode miner\footnote{http://adrem.ua.ac.be/mining-closed-strict-episodes}. It allowed us to verify the correctness of our implementation. The closed episode miner differs in a few ways from our implementation:

\begin{itemize}
\item It mines general episodes, not just parallel and serial episodes.
\item It mines closed episodes. An episode is closed if there are no superepisodes of the same frequency. Mining only closed episodes helps reduce the amount of output, by excluding non-closed episodes. The closed episode miner has options to mine non-closed episodes as well, which allows us to compare the output of both implementations.
\item While our implementation finds episodes of one class at a time, as specified by the user --- parallel or serial --- the closed episode miner finds episodes of all classes at once --- parallel, serial and general.
\end{itemize}

Given the above, in order to compare the two implementations in terms of their output, we have to enable the options that cause the closed episode miner to find non-closed episodes, since our implementation does not exclude non-closed episodes; and we have to filter out those episodes which don't match the class of episodes we're currently mining.

\iffalse
Mining general episodes is of higher complexity than parallel and serial episodes.

Parallel episodes can only ``grow'' by adding a new node, and serial episodes grow in much the same way: by adding a node and an edge at the same time. Pattern explosion is a bigger concern with general episodes: additionally, they can grow by adding only an edge. So, ideally, our implementation should be faster than the closed episode miner.

% TODO include this somehow?
\fi

We validated the output of our implementation with the output of the closed episode miner, running the closed episode miner alongside our implementation on the same datasets and parameter configurations, then parsing and comparing the results. Thanks to the closed episode miner we were able to eliminate quite a few errors in our implementation.

We did find, however, a possible bug in the closed episode miner. When mining parallel episodes in the example sequence (Figure~\ref{fig:event-sequence}), using the weighted-window frequency and with a window width of 8, the frequencies for a few episodes differed, as shown in Table~\ref{table:closepi-frequency-difference}. Observing the sequence, each of these episodes has two overlapping minimal windows, of which the second one has a greater weight. Our implementation seems to correctly select the window with the higher weight, while the closed episode miner seems to choose the first window.

Further distilling the issue, we tested a simpler example: the sequence $ \langle (a, 1),\allowbreak(a, 3),\allowbreak(a, 4) \rangle $, episode $ \{ a, a \} $, and a window width of at least 3. The weighted-window frequency is clearly $ 1 / 2 $, being the weight of the minimal window $ [3, 5) $, but the closed episode miner reported $ 1 / 3 $, so it indisputably selected $ [1, 4) $.

\begin{table}
\centering

\begin{tabulary}{\textwidth}{ C|C|C }

$ \alpha $ & $ fr_w(\alpha) $ (ours) & $ fr_w(\alpha) $ (closed episode miner) \\
\hline
$ \{ b, d \} $ & $ 0.2 $ ($ 1/5 $) & $ 0.166 \ldots $ ($ 1/6 $) \\
$ \{ a, b, d \} $ & $ 0.2 $ ($ 1/5 $) & $ 0.142857 \ldots $ ($ 1/7 $) \\
$ \{ a, b, e \} $ & $ 0.25 $ ($ 1/4 $) & $ 1.66 \ldots $ ($ 1/6 $) \\

\end{tabulary}

\caption{Differing weighted-window frequency values between two implementations, mining the example sequence from Figure~\ref{fig:event-sequence}.}
\label{table:closepi-frequency-difference}
\end{table}

\section{Datasets}

We will conduct experiments using a number of datasets:

\begin{itemize}
\item \emph{abstract}: a dataset consisting of the first 739 NSF award abstracts from 1990, merged into one long sequence\footnote{\url{http://kdd.ics.uci.edu/databases/nsfabs/nsfawards.html}}.
\item \emph{tolstoy}: Leo Tolstoy's novel Anna Karenina, from Project Gutenberg\footnote{\url{https://www.gutenberg.org/ebooks/1399}}.
\item \emph{trains}: a dataset consisting of departure times of delayed trains in a Belgian railway station, for trains with a delay of at least three minutes. This data is anonymized, so we won't be able derive any meaning from the patterns, but it it an interesting dataset nonetheless, because contrary to the textual datasets, \emph{trains} contains sparse, real-time data. The time interval between two subsequent timestamps represents one second.
\end{itemize}

The textual datasets were preprocessed by lemmatizing using the Porter stemmer\footnote{\url{https://tartarus.org/martin/PorterStemmer}} and stop words were removed.

Table~\ref{table:datasets-numbers} shows some statistics about each dataset, where $ | \Sigma | $ is the size of the alphabet, $ | s | $ is the number of events, and $ T_e - T_s $ is the time range of the sequence. For dense sequences, $ T_e - T_s = | s | $. For sparse sequences, a window of $ \rho $ contains on average $ | s | \frac\rho{T_e - T_s} $ events.

\begin{table}
\centering

\begin{tabulary}{\textwidth}{ L|RRRC }

dataset & \multicolumn{1}{c}{$ | \Sigma | $} & \multicolumn{1}{c}{$ | s | $} & \multicolumn{1}{c}{$ T_e - T_s $} & type \\
\hline
\emph{abstract} & $ 51\,346 $ & $ 67\,828 $ & $ 67\,828 $ & dense \\
\emph{tolstoy} & $ 95\,623 $ & $ 124\,627 $ & $ 124\,627 $ & dense \\
\emph{trains} & $ 7\,874 $ & $ 10\,115 $ & $ 26\,626\,667 $ & sparse \\

\end{tabulary}

\caption{Some properties of the datasets $ (s, T_s, T_e) $.}
\label{table:datasets-numbers}
\end{table}

We see that all of the datasets have quite large alphabets. Figure~\ref{fig:alphabet-frequencies} shows the number of occurrences of the most frequent event types, ordered by frequency, for each of the datasets. We observe a long tail\footnote{\url{https://en.wikipedia.org/wiki/Long_tail}} for \emph{abstract} and \emph{tolstoy}, which is common for natural-language texts. In \emph{trains} there is a similar progression, with a small number of the event types occurring a significant number of times, and the vast majority of event types occurring very rarely. Note that the graphs don't even show all event types, although for \emph{abstract} and \emph{trains} the rightmost event types occur only once, and those for \emph{tolstoy} occur fewer than 10 times.

\begin{figure}

\begin{subfigure}[b]{0.48\textwidth}
\centering

\begin{tikzpicture}[scale=0.65]

\begin{axis}[
    xlabel={event types ordered by frequency},
    ylabel={number of occurrences},
    % ymode=log
]

\addplot table [x=rank,y=count,mark=none] {experiments/nsf-alphabet-frequency-300.dat};

\end{axis}

\end{tikzpicture}

\caption{The frequency of the 300 most frequent events in \emph{abstract}.}
\label{fig:frequency-plot-nsf}
\end{subfigure}\hfill%
\begin{subfigure}[b]{0.48\textwidth}
\centering

\begin{tikzpicture}[scale=0.65]

\begin{axis}[
    xlabel={event types ordered by frequency},
    ylabel={number of occurrences},
    % ymode=log
]

\addplot table [x=rank,y=count,mark=none] {experiments/tolstoy-alphabet-frequency-2200.dat};

\end{axis}

\end{tikzpicture}

\caption{The frequency of the 2200 most frequent events in \emph{tolstoy}.}
\label{fig:frequency-plot-tolstoy}
\end{subfigure}

\par\bigskip

\begin{subfigure}[b]{\textwidth}
\centering

\begin{tikzpicture}[scale=0.65]

\begin{axis}[
    xlabel={event types ordered by frequency},
    ylabel={number of occurrences},
    % ymode=log
]

\addplot table [x=rank,y=count,mark=none] {experiments/trains-alphabet-frequency-300.dat};

\end{axis}

\end{tikzpicture}

\caption{The frequency of the 300 most frequent events in \emph{trains}.}
\label{fig:frequency-plot-trains}
\end{subfigure}

\caption{The frequency of the most frequent event types in the datasets.}
\label{fig:alphabet-frequencies}
\end{figure}


\section{Performance}
\label{sec:performance}

To assess the efficiency of the algorithm, we inspect the runtime for different input parameters: comparing both episode classes, the frequency measures, while varying
% the window width and
the frequency and confidence thresholds.

The performance experiments were ran as follows. For specified episode classes, frequency measures, a list of window widths, and a range of frequency thresholds, an experiment would run the Cartesian product of all these parameters, within time and memory constraints. A few particularities:

\begin{itemize}
\item The range of frequency thresholds has an exponentially decreasing nature: it is specified by a (high) starting threshold, a multiplier $ \in (0, 1) $, and a lower bound. Each iteration, the current frequency threshold is multiplied with the multiplier to obtain the next frequency threshold. For example, with a multiplier of 0.90, the next threshold is always 10\% smaller than the last.
\item For each combination of episode class, frequency measure, and window width, a thread is run with progressively lower frequency thresholds, as described above. If memory runs out or the timeout is exceeded before reaching the lower bound, all lower frequency thresholds for that combination of episode class, frequency measure, and window width are skipped, as they will take at least as much time and memory as the current threshold.
\end{itemize}

All performance experiments were run on the same machine; the full specifications of which can be found online\footnote{The specifications can be found at \url{https://support.apple.com/kb/sp623}. 2.7 GHz model; memory manually upgraded to 12~GB. Running macOS 10.13.5. All C++ code compiled with clang-900.0.39.2 from LLVM~9.0. All Java code run with Java~SE~1.8.}.


\subsection{Episodes}
\label{sec:performance-episodes}

\begin{figure}

\begin{subfigure}[b]{0.5\textwidth}
\centering

\begin{tikzpicture}[scale=0.65]

\begin{axis}[
    legend entries={fixed windows,minimal windows,weighted windows},
    legend style={legend pos=south east},
    xlabel={number of frequent episodes},
    ylabel={runtime (s)},
    xmode=log,
    ymode=log,
]

\addplot table [x=num-frequent-episodes,y=duration-s] {experiments/nsf/nsf-parallel-fixed-windows-8.tsv};
\addplot table [x=num-frequent-episodes,y=duration-s] {experiments/nsf/nsf-parallel-minimal-windows-8.tsv};
\addplot table [x=num-frequent-episodes,y=duration-s] {experiments/nsf/nsf-parallel-weighted-windows-8.tsv};

\end{axis}

\end{tikzpicture}

\caption{parallel episodes}
\end{subfigure}%
\begin{subfigure}[b]{0.5\textwidth}
\centering

\begin{tikzpicture}[scale=0.65]

\begin{axis}[
    legend entries={fixed windows,minimal windows,weighted windows},
    legend style={legend pos=south east},
    xlabel={number of frequent episodes},
    ylabel={runtime (s)},
    xmode=log,
    ymode=log,
]

\addplot table [x=num-frequent-episodes,y=duration-s] {experiments/nsf/nsf-serial-fixed-windows-8.tsv};
\addplot table [x=num-frequent-episodes,y=duration-s] {experiments/nsf/nsf-serial-minimal-windows-8.tsv};
\addplot table [x=num-frequent-episodes,y=duration-s] {experiments/nsf/nsf-serial-weighted-windows-8.tsv};

\end{axis}

\end{tikzpicture}

\caption{serial episodes}
\end{subfigure}

\caption{Runtimes for finding episodes in dataset \emph{abstract} using a window width of 8.}
\label{fig:runtimes-nsf-8}
\end{figure}

\begin{figure}

\begin{subfigure}[b]{0.5\textwidth}
\centering

\begin{tikzpicture}[scale=0.65]

\begin{axis}[
    legend entries={fixed windows,minimal windows,weighted windows},
    legend style={legend pos=south east},
    xlabel={number of frequent episodes},
    ylabel={runtime (s)},
    xmode=log,
    ymode=log,
]

\addplot table [x=num-frequent-episodes,y=duration-s] {experiments/tolstoy/tolstoy-parallel-fixed-windows-15.tsv};
\addplot table [x=num-frequent-episodes,y=duration-s] {experiments/tolstoy/tolstoy-parallel-minimal-windows-15.tsv};
\addplot table [x=num-frequent-episodes,y=duration-s] {experiments/tolstoy/tolstoy-parallel-weighted-windows-15.tsv};

\end{axis}

\end{tikzpicture}

\caption{parallel episodes}
\end{subfigure}%
\begin{subfigure}[b]{0.5\textwidth}
\centering

\begin{tikzpicture}[scale=0.65]

\begin{axis}[
    legend entries={fixed windows,minimal windows,weighted windows},
    legend style={legend pos=north west},
    xlabel={number of frequent episodes},
    ylabel={runtime (s)},
    xmode=log,
    ymode=log,
]

\addplot table [x=num-frequent-episodes,y=duration-s] {experiments/tolstoy/tolstoy-serial-fixed-windows-15.tsv};
\addplot table [x=num-frequent-episodes,y=duration-s] {experiments/tolstoy/tolstoy-serial-minimal-windows-15.tsv};
\addplot table [x=num-frequent-episodes,y=duration-s] {experiments/tolstoy/tolstoy-serial-weighted-windows-15.tsv};

\end{axis}

\end{tikzpicture}

\caption{serial episodes}
\end{subfigure}

\caption{Runtimes for finding episodes in dataset \emph{tolstoy} using a window width of 15.}
\label{fig:runtimes-tolstoy-15}
\end{figure}

\begin{figure}

\begin{subfigure}[b]{0.5\textwidth}
\centering

\begin{tikzpicture}[scale=0.65]

\begin{axis}[
    legend entries={fixed windows,minimal windows,weighted windows},
    legend style={legend pos=south east},
    xlabel={number of frequent episodes},
    ylabel={runtime (s)},
    xmode=log,
    ymode=log,
]

\addplot table [x=num-frequent-episodes,y=duration-s] {experiments/trains/trains-parallel-fixed-windows-900.tsv};
\addplot table [x=num-frequent-episodes,y=duration-s] {experiments/trains/trains-parallel-minimal-windows-900.tsv};
\addplot table [x=num-frequent-episodes,y=duration-s] {experiments/trains/trains-parallel-weighted-windows-900.tsv};

\end{axis}

\end{tikzpicture}

\caption{parallel episodes}
\end{subfigure}%
\begin{subfigure}[b]{0.5\textwidth}
\centering

\begin{tikzpicture}[scale=0.65]

\begin{axis}[
    legend entries={fixed windows,minimal windows,weighted windows},
    legend style={legend pos=south east},
    xlabel={number of frequent episodes},
    ylabel={runtime (s)},
    xmode=log,
    ymode=log,
]

\addplot table [x=num-frequent-episodes,y=duration-s] {experiments/trains/trains-serial-fixed-windows-900.tsv};
\addplot table [x=num-frequent-episodes,y=duration-s] {experiments/trains/trains-serial-minimal-windows-900.tsv};
\addplot table [x=num-frequent-episodes,y=duration-s] {experiments/trains/trains-serial-weighted-windows-900.tsv};

\end{axis}

\end{tikzpicture}

\caption{serial episodes}
\end{subfigure}

\caption{Runtimes for finding episodes in dataset \emph{trains} using a window width of 900.}
\label{fig:runtimes-trains-900}
\end{figure}

We would like to compare the efficiency for the different frequency measures across a range of frequency thresholds. However we should not evaluate the runtimes as a function of the frequency threshold directly, since the values for each of the measures are semantically different. For instance, the weighted-window frequency of an episode in a sequence is at most equal to the disjoint-window frequency; so for otherwise equal parameters (including thresholds), the weighted-window frequency will produce a subset of the disjoint-window frequency. Instead we can compare runtimes as a function of the number of frequent episodes.

The graph in Figure~\ref{fig:runtimes-nsf-8} shows runtimes for mining episodes from \emph{abstract}. We see that, for parallel episodes, the fixed-window frequency takes significantly less time to generate a given amount of episodes than the other measures. The disjoint-window frequency and the weighted-window frequency are very close to each other until approximately one thousand episodes are found; then they start to diverge, and the the weighted-window frequency takes significantly more time than the other two.

For serial episodes, the fixed-window frequency has much less of an advantage. In fact, the disjoint-window frequency seems to overtake the fixed-window frequency for large amounts of episodes produced. This could be explained by the fact that the data pass algorithm that finds minimal windows of serial episodes is slightly simpler than the algorithm that determines the fixed-window frequency of serial episodes. Again the weighted-window frequency is slower, but the difference is smaller this time. Also note that for all measures, runtimes for serial episodes were generally higher --- since the experiments were time-constrained, the graph stops around the order of $ 10^5 $ for parallel episodes, and around $ 10^3 $ for serial episodes.

Figure~\ref{fig:runtimes-tolstoy-15} shows the results for \emph{tolstoy}. Though \emph{tolstoy} is approximately twice as long as \emph{abstract}, the progression for both serial and parallel is very similar.

The results for the sparse dataset \emph{trains} are a bit different (Figure~\ref{fig:runtimes-trains-900}). For parallel episodes, the fixed-window frequency still has the advantage. Curiously, the time consumption for the weighted-window frequency increases steadily, until around $ 10^3 $ episodes or so, when it plateaus, and eventually matches the disjoint-window frequency. For serial episodes, the weighted-window frequency shows a similar plateau, after which it even beats the fixed-window frequency and the disjoint-window frequency for some runs. We have no explanation for this interesting behaviour.

Note that significantly more episodes are found in the \emph{trains} dataset --- up to the order of $ 10^6 $. This is most likely due to the fact that \emph{trains} is considerably smaller than \emph{abstract} and \emph{tolstoy}, in both alphabet size and number of events. Perhaps, if given more time, the progression would look similar for \emph{abstract} and \emph{tolstoy}.

\begin{figure}

\newcommand\nsfepisodefrequenciesbysizeaxis[1]{%
\begin{axis}[
    legend entries={1-episodes,2-episodes,3-episodes,4-episodes,5-episodes,6-episodes,7-episodes,8-episodes},
    legend style={legend pos=north east,legend style={nodes={scale=0.75}}},
    xlabel={frequency threshold},
    ylabel={number of frequent $ l $-episodes},
    xmode=log,
    ymode=log,
]

\addplot table [x=frequency-threshold,y=num-frequent-1-episodes] {#1};
\addplot table [x=frequency-threshold,y=num-frequent-2-episodes] {#1};
\addplot table [x=frequency-threshold,y=num-frequent-3-episodes] {#1};
\addplot table [x=frequency-threshold,y=num-frequent-4-episodes] {#1};
\addplot table [x=frequency-threshold,y=num-frequent-5-episodes] {#1};
\addplot table [x=frequency-threshold,y=num-frequent-6-episodes] {#1};
\addplot table [x=frequency-threshold,y=num-frequent-7-episodes] {#1};
\addplot table [x=frequency-threshold,y=num-frequent-8-episodes] {#1};

\end{axis}
}

\begin{subfigure}[b]{0.5\textwidth}
\centering
\begin{tikzpicture}[scale=0.65]

\nsfepisodefrequenciesbysizeaxis{experiments/nsf/nsf-parallel-fixed-windows-8.tsv}

\end{tikzpicture}
\caption{\emph{abstract}}
\label{fig:episode-frequencies-by-size-abstract}
\end{subfigure}%
\begin{subfigure}[b]{0.5\textwidth}
\centering
\begin{tikzpicture}[scale=0.65]

\begin{axis}[
    legend entries={1-episodes,2-episodes,3-episodes,4-episodes,5-episodes,6-episodes,7-episodes,8-episodes,9-episodes,10-episodes,11-episodes,12-episodes,13-episodes,14-episodes,15-episodes},
    legend style={legend pos=outer north east,legend style={nodes={scale=0.75}}},
    xlabel={frequency threshold},
    ylabel={number of frequent $ l $-episodes},
    xmode=log,
    ymode=log,
]

\addplot table [x=frequency-threshold,y=num-frequent-1-episodes] {experiments/trains/trains-parallel-fixed-windows-900.tsv};
\addplot table [x=frequency-threshold,y=num-frequent-2-episodes] {experiments/trains/trains-parallel-fixed-windows-900.tsv};
\addplot table [x=frequency-threshold,y=num-frequent-3-episodes] {experiments/trains/trains-parallel-fixed-windows-900.tsv};
\addplot table [x=frequency-threshold,y=num-frequent-4-episodes] {experiments/trains/trains-parallel-fixed-windows-900.tsv};
\addplot table [x=frequency-threshold,y=num-frequent-5-episodes] {experiments/trains/trains-parallel-fixed-windows-900.tsv};
\addplot table [x=frequency-threshold,y=num-frequent-6-episodes] {experiments/trains/trains-parallel-fixed-windows-900.tsv};
\addplot table [x=frequency-threshold,y=num-frequent-7-episodes] {experiments/trains/trains-parallel-fixed-windows-900.tsv};
\addplot table [x=frequency-threshold,y=num-frequent-8-episodes] {experiments/trains/trains-parallel-fixed-windows-900.tsv};
\addplot table [x=frequency-threshold,y=num-frequent-9-episodes] {experiments/trains/trains-parallel-fixed-windows-900.tsv};
\addplot table [x=frequency-threshold,y=num-frequent-10-episodes] {experiments/trains/trains-parallel-fixed-windows-900.tsv};
\addplot table [x=frequency-threshold,y=num-frequent-11-episodes] {experiments/trains/trains-parallel-fixed-windows-900.tsv};
\addplot table [x=frequency-threshold,y=num-frequent-12-episodes] {experiments/trains/trains-parallel-fixed-windows-900.tsv};
\addplot table [x=frequency-threshold,y=num-frequent-13-episodes] {experiments/trains/trains-parallel-fixed-windows-900.tsv};
\addplot table [x=frequency-threshold,y=num-frequent-14-episodes] {experiments/trains/trains-parallel-fixed-windows-900.tsv};
\addplot table [x=frequency-threshold,y=num-frequent-15-episodes] {experiments/trains/trains-parallel-fixed-windows-900.tsv};


\end{axis}

\end{tikzpicture}
\caption{\emph{trains}}
\label{fig:episode-frequencies-by-size-trains}
\end{subfigure}
\caption{Plots of the number of parallel episodes found in \emph{abstract} and \emph{trains} for the fixed-window frequency across different thresholds, episodes grouped by size. Episodes mined from dataset \emph{abstract} using the fixed-window frequency measure.}
\label{fig:episode-frequencies-by-size}
\end{figure}

Besides runtime comparisons, it might also be useful to know how many episodes of each size are produced. As we will expand on later, there is a quality angle to this as well. In Figure~\ref{fig:episode-frequencies-by-size} we see that each time the threshold is lowered by a certain factor, the amount of another size class of episodes rises significantly. For \emph{abstract} we find up to 8-episodes and for \emph{trains} we even find up to 15-episodes.

Plots for the other kinds of episodes and frequency measures showed similar results, though less extreme --- perhaps that can be explained by the fact that all experiments were run under the same time constraints.


\subsection{Association rules}

\begin{figure}

\begin{subfigure}[b]{0.5\textwidth}
\centering

\begin{tikzpicture}[scale=0.65]

\begin{axis}[
    legend entries={fixed windows,minimal windows,weighted windows},
    legend style={legend pos=north west},
    xlabel={number of frequent episodes},
    ylabel={runtime (s)},
    xmode=log,
    ymode=log,
]

\addplot table [x=num-frequent-episodes,y=duration-rules-0.0] {experiments/tolstoy/tolstoy-parallel-fixed-windows-15.tsv};
\addplot table [x=num-frequent-episodes,y=duration-rules-0.0] {experiments/tolstoy/tolstoy-parallel-minimal-windows-15.tsv};
\addplot table [x=num-frequent-episodes,y=duration-rules-0.0] {experiments/tolstoy/tolstoy-parallel-weighted-windows-15.tsv};

\end{axis}

\end{tikzpicture}

\caption{rules consisting of parallel episodes}
\end{subfigure}%
\begin{subfigure}[b]{0.5\textwidth}
\centering

\begin{tikzpicture}[scale=0.65]

\begin{axis}[
    legend entries={fixed windows,minimal windows,weighted windows},
    legend style={legend pos=north west},
    xlabel={number of frequent episodes},
    ylabel={runtime (s)},
    xmode=log,
    ymode=log,
]

\addplot table [x=num-frequent-episodes,y=duration-rules-0.0] {experiments/tolstoy/tolstoy-serial-fixed-windows-15.tsv};
\addplot table [x=num-frequent-episodes,y=duration-rules-0.0] {experiments/tolstoy/tolstoy-serial-minimal-windows-15.tsv};
\addplot table [x=num-frequent-episodes,y=duration-rules-0.0] {experiments/tolstoy/tolstoy-serial-weighted-windows-15.tsv};

\end{axis}

\end{tikzpicture}

\caption{rules consisting of serial episodes}
\end{subfigure}

\caption{Runtimes for finding association rules from episodes generated from \emph{tolstoy} (confidence threshold of 0; same experiment as Figure~\ref{fig:runtimes-tolstoy-15}) as a function of the number of frequent episodes.}
\label{fig:runtimes-rules-tolstoy-15}
\end{figure}

\begin{figure}

\begin{subfigure}[b]{0.5\textwidth}
\centering

\begin{tikzpicture}[scale=0.65]

\begin{axis}[
    legend entries={fixed windows,minimal windows,weighted windows},
    legend style={legend pos=north west},
    xlabel={number of frequent episodes},
    ylabel={runtime (s)},
    xmode=log,
    ymode=log,
]

\addplot table [x=num-frequent-episodes,y=duration-rules-0.9] {experiments/trains/trains-parallel-fixed-windows-900.tsv};
\addplot table [x=num-frequent-episodes,y=duration-rules-0.9] {experiments/trains/trains-parallel-minimal-windows-900.tsv};
\addplot table [x=num-frequent-episodes,y=duration-rules-0.9] {experiments/trains/trains-parallel-weighted-windows-900.tsv};

\end{axis}

\end{tikzpicture}

\caption{rules consisting of parallel episodes}
\label{fig:runtimes-rules-trains-900-parallel}
\end{subfigure}%
\begin{subfigure}[b]{0.5\textwidth}
\centering

\begin{tikzpicture}[scale=0.65]

\begin{axis}[
    legend entries={fixed windows,minimal windows,weighted windows},
    legend style={legend pos=north west},
    xlabel={number of frequent episodes},
    ylabel={runtime (s)},
    xmode=log,
    ymode=log,
]

\addplot table [x=num-frequent-episodes,y=duration-rules-0.9] {experiments/trains/trains-serial-fixed-windows-900.tsv};
\addplot table [x=num-frequent-episodes,y=duration-rules-0.9] {experiments/trains/trains-serial-minimal-windows-900.tsv};
\addplot table [x=num-frequent-episodes,y=duration-rules-0.9] {experiments/trains/trains-serial-weighted-windows-900.tsv};

\end{axis}

\end{tikzpicture}

\caption{rules consisting of serial episodes}
\label{fig:runtimes-rules-trains-900-serial}
\end{subfigure}

\caption{Runtimes for finding association rules from episodes generated from \emph{trains} (confidence threshold of 0.9; same experiment as figure~\ref{fig:runtimes-trains-900}) as a function of the number of frequent episodes.}
\label{fig:runtimes-rules-trains-900}
\end{figure}

As we saw in Algorithm~\ref{alg:association-rules-top-level}, for each frequent episode $ \beta $, all $ \alpha \Rightarrow \beta $ with $ \alpha \subset \beta $ are considered. Assuming $ \beta $ has an injective $ lab $-function (an event type appears at most once), the number of subepisodes of an episode is exponential in the size of the episode; and so the same holds for the number of association rules to be considered.

Figure~\ref{fig:runtimes-rules-tolstoy-15} shows runtimes for generating association rules from \emph{tolstoy} using different episode classes, as a function of the number of frequent episodes. Despite the concerns we expressed in the previous paragraph, the runtime seems to be mostly polynomial --- indicating that the number of frequent episodes remains the most important factor for these experiments. Finding the association rules from parallel episodes in \emph{trains} using the fixed-window frequency started to take significant time for the lowest frequency thresholds, but we think that must be mostly due to the large number of episodes generated.

We mentioned earlier that computing the fixed-window confidence is trivial, given the frequency of the episodes; and yet for some of the experiments here, finding the confident association rules took longer than the weighted-window confidence. This can be explained by the overhead of storing the rules that turned out to be confident, requiring more time if there are  more confident rules.

\iffalse
\subsection{Number of candidate episodes versus number of frequent episodes}

% --- Plotting the number of candidates and the number of frequent episodes. Maybe not so useful.
\begin{tikzpicture}

\begin{axis}[
    legend entries={fixed windows,minimal windows,weighted windows},
    legend style={legend pos=outer north east},
    xlabel={number of candidates},
    ylabel={number of frequent episodes},
]

\addplot table [x=num-frequent-episodes,y=num-candidates] {experiments/nsf/nsf-parallel-fixed-windows-8.tsv};
\addplot table [x=num-frequent-episodes,y=num-candidates] {experiments/nsf/nsf-parallel-minimal-windows-8.tsv};
\addplot table [x=num-frequent-episodes,y=num-candidates] {experiments/nsf/nsf-parallel-weighted-windows-8.tsv};

\end{axis}

\end{tikzpicture}

\begin{tikzpicture}

\begin{axis}[
    legend entries={fixed windows,minimal windows,weighted windows},
    legend style={legend pos=outer north east},
    xlabel={number of candidates},
    ylabel={number of frequent episodes},
]

\addplot table [x=num-frequent-episodes,y=percentage-frequent-of-candidates] {experiments/nsf/nsf-parallel-fixed-windows-8.tsv};
\addplot table [x=num-frequent-episodes,y=percentage-frequent-of-candidates] {experiments/nsf/nsf-parallel-minimal-windows-8.tsv};
\addplot table [x=num-frequent-episodes,y=percentage-frequent-of-candidates] {experiments/nsf/nsf-parallel-weighted-windows-8.tsv};

\end{axis}

\end{tikzpicture}

\begin{tikzpicture}

\begin{axis}[
    legend entries={fixed windows,minimal windows,weighted windows},
    legend style={legend pos=outer north east},
    xlabel={number of candidates},
    ylabel={number of frequent episodes},
]

\addplot table [x=num-candidates,y=num-frequent-episodes] {experiments/trains/trains-parallel-fixed-windows-900.tsv};
\addplot table [x=num-candidates,y=num-frequent-episodes] {experiments/trains/trains-parallel-minimal-windows-900.tsv};
\addplot table [x=num-candidates,y=num-frequent-episodes] {experiments/trains/trains-parallel-weighted-windows-900.tsv};

\end{axis}

\end{tikzpicture}
% --- ---

The database passes are an expensive --- if not the most expensive --- part of the mining algorithm. Plotting the number of frequent episodes in the output and the number of candidates that were considered in a pass over the sequence gives us an idea of the amount of internal work the algorithms deal with for the output that gets produced.
\fi



\subsection{Comparing sequence lengths}

\begin{figure}

\begin{subfigure}[b]{0.5\textwidth}
\centering

\begin{tikzpicture}[scale=0.65]

\begin{axis}[
    legend pos=north west,
    legend entries={fixed windows,minimal windows,weighted windows},
    xlabel={portion of the whole sequence},
    ylabel={runtime (s)},
    % ymode=log,
]

\addplot table [x=portion,y=duration-s] {experiments/length/tolstoy-lengths-parallel-fixed-windows.dat};
\addplot table [x=portion,y=duration-s] {experiments/length/tolstoy-lengths-parallel-minimal-windows.dat};
\addplot table [x=portion,y=duration-s] {experiments/length/tolstoy-lengths-parallel-weighted-windows.dat};

\end{axis}

\end{tikzpicture}
\caption{linear scale}
\label{fig:tolstoy-runtime-vs-length-linear-scale}
\end{subfigure}%
\begin{subfigure}[b]{0.5\textwidth}
\centering
\begin{tikzpicture}[scale=0.65]

\begin{axis}[
    legend pos=south east,
    legend entries={fixed windows,minimal windows,weighted windows},
    xlabel={portion of the whole sequence},
    ylabel={runtime (s)},
    xmode=log,
    ymode=log,
]

\addplot table [x=portion,y=duration-s] {experiments/length/tolstoy-lengths-parallel-fixed-windows.dat};
\addplot table [x=portion,y=duration-s] {experiments/length/tolstoy-lengths-parallel-minimal-windows.dat};
\addplot table [x=portion,y=duration-s] {experiments/length/tolstoy-lengths-parallel-weighted-windows.dat};

% \addplot table [x=portion,y=duration-s] {experiments/length/tolstoy-lengths-serial-fixed-windows.dat};
% \addplot table [x=portion,y=duration-s] {experiments/length/tolstoy-lengths-serial-minimal-windows.dat};
% \addplot table [x=portion,y=duration-s] {experiments/length/tolstoy-lengths-serial-weighted-windows.dat};

\end{axis}

\end{tikzpicture}
\caption{log--log scale}
\label{fig:tolstoy-runtime-vs-length-log-scale}
\end{subfigure}

\caption{Runtimes for different portions of \emph{tolstoy} using a fixed frequency threshold for each frequency measure. $ \rho = 15 $.}
\label{fig:tolstoy-runtime-vs-length}
\end{figure}

In sequential pattern mining, sequences are expected to be very long. Therefore we would like to have an idea of how the length of the sequence affects the performance. We measured the runtime of the algorithm for different-sized prefixes of the \emph{tolstoy} dataset for all episode classes and frequency measures. The results for parallel episodes can be seen in Figure~\ref{fig:tolstoy-runtime-vs-length}. The results for serial episodes were similar.

An important note for the interpretation of these plots: all the runs for a given measure used equal frequency thresholds, but the thresholds for each of the measures were chosen independently. They were chosen such that the runtime of each experiment was at most 30 seconds; that's why the data points meet up at the full sequence. So only the progressions are important here.

The general progression is fairly similar for all of the measures, although the minimal-windows-based measures rise quicker than the fixed-window frequency. While unfortunately the complexity is worse than linear (Figure~\ref{fig:tolstoy-runtime-vs-length-linear-scale}), judging by the mostly straight lines in the log--log plot (Figure~\ref{fig:tolstoy-runtime-vs-length-log-scale}) it does seem to be polynomial.

\subsection{Comparing performance with the closed episode miner}

Alongside the performance experiments, we ran the closed episode miner on some of the same parameter configurations.

[TODO write results]

% TODO do some runs and write about them (refer back to earlier experiments)



\section{Quality}

In this section, we will do a qualitative analysis of the output of our algorithm. We compare the top-ranked results across the different frequency measures mutually, and to those of cohesion-based interestingness measures, which we will clarify later.

\iffalse
\subsection{Comparing the frequency measures on a toy example}

We consider the example sequence of the example in Figure~\ref{fig:event-sequence}, which was used as an example throughout chapter~\ref{sec:problem-statement}. A small sequence is interesting to analyze because we have a full overview of the dataset, and can therefore provide insight into how the frequency and confidence values came to be. Also, it is possible to generate all episodes that cover the sequence for a certain window size, using a low frequency threshold.

We'll generate all episodes using all of the frequency measures we implemented.
\fi

% TODO not do?

\subsection{Analysis of episodes mined from \emph{tolstoy} dataset}
\label{sec:experiments-quality-episodes}

\begin{table}
\begin{tabulary}{\textwidth}{R|L|L|L}%
\# & fixed-window fr. & disjoint-window fr. & weighted-window fr. \\
\hline
1 & $ \{ \text{levin} \} $ (20913) & $ \{ \text{levin} \} $ (1629) & $ \{ \text{levin} \} $ (1629) \\
2 & $ \{ \text{vronski} \} $ (11165) & $ \{ \text{vronski} \} $ (865) & $ \{ \text{vronski} \} $ (865) \\
3 & $ \{ \text{anna} \} $ (10699) & $ \{ \text{anna} \} $ (823) & $ \{ \text{anna} \} $ (823) \\
4 & $ \{ \text{thought} \} $ (8994) & $ \{ \text{kitti} \} $ (672) & $ \{ \text{kitti} \} $ (672) \\
5 & $ \{ \text{time} \} $ (8948) & $ \{ \text{thought} \} $ (663) & $ \{ \text{thought} \} $ (663) \\
6 & $ \{ \text{kitti} \} $ (8826) & $ \{ \text{time} \} $ (651) & $ \{ \text{time} \} $ (651) \\
7 & $ \{ \text{hand} \} $ (8645) & $ \{ \text{hand} \} $ (651) & $ \{ \text{hand} \} $ (651) \\
8 & $ \{ \text{alexei} \} $ (8619) & $ \{ \text{smile} \} $ (632) & $ \{ \text{smile} \} $ (632) \\
9 & $ \{ \text{smile} \} $ (8549) & $ \{ \text{alexei} \} $ (632) & $ \{ \text{alexei} \} $ (632) \\
10 & $ \{ \text{face} \} $ (8315) & $ \{ \text{face} \} $ (598) & $ \{ \text{face} \} $ (598) \\
11 & $ \{ \text{ey} \} $ (8062) & $ \{ \text{love} \} $ (595) & $ \{ \text{love} \} $ (595) \\
12 & $ \{ \text{alexandrovitch} \} $ (7842) & $ \{ \text{alexandrovitch} \} $ (571) & $ \{ \text{alexandrovitch} \} $ (571) \\
13 & $ \{ \text{felt} \} $ (7753) & $ \{ \text{alexei},\allowbreak\text{alexandrovitch} \} $ (571) & $ \{ \text{ey} \} $ (570) \\
14 & $ \{ \text{man} \} $ (7751) & $ \{ \text{ey} \} $ (570) & $ \{ \text{man} \} $ (565) \\
15 & $ \{ \text{feel} \} $ (7596) & $ \{ \text{man} \} $ (565) & $ \{ \text{feel} \} $ (561) \\
\end{tabulary}%
\caption{The top 15 parallel episodes found by our algorithm, with $ \rho = 15 $, and for the three frequency measures.}
\label{table:fmw-tolstoy-top-15-parallel-episodes}
\end{table}

% TODO elaborate on anti-monotonic frequency measures disadvantageing larger episodes, etc
\iffalse
As we saw in Section~\ref{sec:experiments-quality-episodes}, anti-monotonic frequency measures inherently put larger episodes at a disadvantage, since an episode is never more frequent than any of its subepisodes. This has a few effects:
\begin{itemize}
\item $ (l + 1) $-episodes will generally score lower than $ l $-episodes, so larger episodes will tend to be ranked lower than smalles ones.
\item As a consequence of the preceding, and because we mine according to a fixed frequency threshold, the number of $ l $-episodes found decreases strongly as $ l $ grows.
\end{itemize}
We can clearly see the second effect if we plot the number of frequent episodes grouped by size: Figure~\ref{fig:episode-frequencies-by-size} gives the numbers for one of the experiments performed in section~\ref{sec:performance-episodes}. We do see, however, that it helps to choose thresholds as low as possible, since the number greater-than-1-episodes rises significantly near the lowest thresholds. Still, larger episodes generally won't score highly compared to their subepisodes, and so they won't do well when episodes are ranked.
\fi

As we would expect, if we rank the output by frequency, the top contains mostly just episodes of size 1 (Table~\ref{table:fmw-tolstoy-top-15-parallel-episodes}). It does give us some information about the text, though not much more than when we simply count the occurrences of all words.

\begin{itemize}
\item We learn of many characters' names --- either first or last, but we don't know many characters' first and last name.
\item Looking at the column for the disjoint-window frequency, we see that two names are mentioned together frequently: $ \{ \text{alexei}, \text{alexandrovitch} \} $. From this information it is likely that a character named \emph{Alexei Alexandrovitch} appears often in the book. (This is indeed the case.) Moreover, we see that $ \{ \text{alexei}, \text{alexandrovitch} \} $ is just as frequent as subepisode $ \{ \text{alexandrovitch} \} $. So wherever the last name \emph{Alexandrovitch} is mentioned, the first name \emph{Alexei} is mentioned nearby (within at most 15 words).
\item Common words like \emph{thought}, \emph{smile}, \emph{face}, \emph{love}, \emph{eye} (stemmed to \emph{ey}), \emph{feel} can give some indication of genre. At least it seems clear that the sequence does not represent a research paper in computer science.

\end{itemize}

\begin{table}
\begin{tabulary}{\textwidth}{R|L|L|L}%
\# & fixed-window fr. & disjoint-window fr. & weighted-window fr. \\
\hline
1 & $ \{ \text{alexei},\allowbreak\text{alexandrovitch} \} $ (7416) & $ \{ \text{alexei},\allowbreak\text{alexandrovitch} \} $ (571) & $ \{ \text{alexei},\allowbreak\text{alexandrovitch} \} $ (286) \\
2 & $ \{ \text{stepan},\allowbreak\text{arkadyevitch} \} $ (7117) & $ \{ \text{stepan},\allowbreak\text{arkadyevitch} \} $ (547) & $ \{ \text{stepan},\allowbreak\text{arkadyevitch} \} $ (274) \\
3 & $ \{ \text{sergei},\allowbreak\text{ivanovitch} \} $ (3763) & $ \{ \text{levin},\allowbreak\text{levin} \} $ (395) & $ \{ \text{sergei},\allowbreak\text{ivanovitch} \} $ (146) \\
4 & $ \{ \text{levin},\allowbreak\text{levin} \} $ (3348) & $ \{ \text{sergei},\allowbreak\text{ivanovitch} \} $ (291) & $ \{ \text{darya},\allowbreak\text{alexandrovna} \} $ (102) \\
5 & $ \{ \text{darya},\allowbreak\text{alexandrovna} \} $ (2739) & $ \{ \text{darya},\allowbreak\text{alexandrovna} \} $ (205) & $ \{ \text{levin},\allowbreak\text{levin} \} $ (61.6) \\
6 & $ \{ \text{levin},\allowbreak\text{kitti} \} $ (2039) & $ \{ \text{levin},\allowbreak\text{kitti} \} $ (202) & $ \{ \text{lidia},\allowbreak\text{ivanovna} \} $ (54) \\
7 & $ \{ \text{anna},\allowbreak\text{vronski} \} $ (1942) & $ \{ \text{stepan},\allowbreak\text{levin} \} $ (199) & $ \{ \text{anna},\allowbreak\text{vronski} \} $ (45.1) \\
8 & $ \{ \text{arkadyevitch},\allowbreak\text{levin} \} $ (1896) & $ \{ \text{arkadyevitch},\allowbreak\text{levin} \} $ (197) & $ \{ \text{smile},\allowbreak\text{levin} \} $ (41.3) \\
9 & $ \{ \text{stepan},\allowbreak\text{levin} \} $ (1887) & $ \{ \text{stepan},\allowbreak\text{arkadyevitch},\allowbreak\text{levin} \} $ (195) & $ \{ \text{levin},\allowbreak\text{kitti} \} $ (41.1) \\
10 & $ \{ \text{stepan},\allowbreak\text{arkadyevitch},\allowbreak\text{levin} \} $ (1784) & $ \{ \text{vronski},\allowbreak\text{vronski} \} $ (191) & $ \{ \text{room},\allowbreak\text{draw} \} $ (40.6) \\
11 & $ \{ \text{smile},\allowbreak\text{levin} \} $ (1778) & $ \{ \text{anna},\allowbreak\text{vronski} \} $ (180) & $ \{ \text{countess},\allowbreak\text{lidia} \} $ (39.4) \\
12 & $ \{ \text{vronski},\allowbreak\text{vronski} \} $ (1722) & $ \{ \text{smile},\allowbreak\text{levin} \} $ (171) & $ \{ \text{thought},\allowbreak\text{levin} \} $ (39.3) \\
13 & $ \{ \text{time},\allowbreak\text{levin} \} $ (1597) & $ \{ \text{anna},\allowbreak\text{anna} \} $ (170) & $ \{ \text{love},\allowbreak\text{love} \} $ (38.2) \\
14 & $ \{ \text{brother},\allowbreak\text{levin} \} $ (1558) & $ \{ \text{time},\allowbreak\text{levin} \} $ (159) & $ \{ \text{arkadyevitch},\allowbreak\text{levin} \} $ (37.2) \\
15 & $ \{ \text{anna},\allowbreak\text{anna} \} $ (1531) & $ \{ \text{good},\allowbreak\text{levin} \} $ (153) & $ \{ \text{agafea},\allowbreak\text{mihalovna} \} $ (37) \\
\end{tabulary}%
\caption{The top 15 parallel episodes found by our algorithm, excluding 1-episodes, with $ \rho = 15 $, and for the three frequency measures.}
\label{table:fmw-tolstoy-top-15-parallel->1-episodes}
\end{table}

\begin{table}

\begin{tabulary}{\textwidth}{R|L|L|L}

\# & disjoint-window fr. & weighted-window fr. \\
\hline
1 & $ \text{alexei} \to \text{alexandrovitch} $ (7401) & $ \text{alexei} \to \text{alexandrovitch} $ (571) & $ \text{alexei} \to \text{alexandrovitch} $ (286) \\
2 & $ \text{stepan} \to \text{arkadyevitch} $ (7106) & $ \text{stepan} \to \text{arkadyevitch} $ (547) & $ \text{stepan} \to \text{arkadyevitch} $ (274) \\
3 & $ \text{sergei} \to \text{ivanovitch} $ (3758) & $ \text{levin} \to \text{levin} $ (395) & $ \text{sergei} \to \text{ivanovitch} $ (146) \\
4 & $ \text{levin} \to \text{levin} $ (3348) & $ \text{sergei} \to \text{ivanovitch} $ (291) & $ \text{darya} \to \text{alexandrovna} $ (102) \\
5 & $ \text{darya} \to \text{alexandrovna} $ (2734) & $ \text{darya} \to \text{alexandrovna} $ (205) & $ \text{levin} \to \text{levin} $ (61.6) \\
6 & $ \text{vronski} \to \text{vronski} $ (1722) & $ \text{vronski} \to \text{vronski} $ (191) & $ \text{lidia} \to \text{ivanovna} $ (54) \\
7 & $ \text{anna} \to \text{anna} $ (1531) & $ \text{anna} \to \text{anna} $ (170) & $ \text{draw} \to \text{room} $ (40.5) \\
8 & $ \text{lidia} \to \text{ivanovna} $ (1438) & $ \text{alexandrovitch} \to \text{alexei} $ (145) & $ \text{countess} \to \text{lidia} $ (39.1) \\
9 & $ \text{love} \to \text{love} $ (1411) & $ \text{kitti} \to \text{kitti} $ (144) & $ \text{love} \to \text{love} $ (38.2) \\
10 & $ \text{arkadyevitch} \to \text{levin} $ (1245) & $ \text{love} \to \text{love} $ (140) & $ \text{agafea} \to \text{mihalovna} $ (37) \\
11 & $ \text{kitti} \to \text{kitti} $ (1200) & $ \text{arkadyevitch} \to \text{levin} $ (137) & $ \text{levin} \to \text{felt} $ (32) \\
12 & $ \text{levin} \to \text{kitti} $ (1160) & $ \text{levin} \to \text{kitti} $ (136) & $ \text{vronski} \to \text{vronski} $ (31.3) \\
13 & $ \text{vronski} \to \text{anna} $ (1137) & $ \text{stepan} \to \text{levin} $ (132) & $ \text{good} \to \text{humor} $ (29.1) \\
14 & $ \text{levin} \to \text{felt} $ (1130) & $ \text{stepan} \to \text{arkadyevitch} \to \text{levin} $ (132) & $ \text{anna} \to \text{arkadyevna} $ (28.5) \\
15 & $ \text{draw} \to \text{room} $ (1126) & $ \text{kitti} \to \text{levin} $ (131) & $ \text{vronski} \to \text{anna} $ (28.1) \\

\end{tabulary}

\caption{The top 15 serial episodes found by our algorithm, excluding 1-episodes, with $ \rho = 15 $, and for the three frequency measures.}
\label{table:fmw-tolstoy-top-15-serial->1-episodes}
\end{table}

From studying Table~\ref{table:fmw-tolstoy-top-15-parallel-episodes} it is clear that simply ranking episodes by frequency is not a good strategy for getting the most out of our algorithm. We should at least filter out the 1-episodes, as those don't give any more information than counting the support of each word that appears in the text. Table~\ref{table:fmw-tolstoy-top-15-parallel->1-episodes} and Table~\ref{table:fmw-tolstoy-top-15-serial->1-episodes} show the rankings of greater-than-1-episodes, for parallel and serial episodes respectively. If we want to highlight even larger episodes, we can group episodes by size.

After filtering out 1-episodes we learn some more things:
\begin{itemize}
\item We find full names --- Alexei Alexandrovitch, Stepan Arkadyevitch, Sergei Ivanovitch, Lidia Ivanovna, Darya Alexandrovna, Agafea Mihalovna --- all are characters' full names. With serial episodes, we find their order as well --- \emph{alexei} usually precedes \emph{alexandrovitch} closely, so $ \text{alexei} \to \text{alexandrovitch} $ is rated more highly than $ \text{alexandrovitch} \to \text{alexei} $.
\item We also find the most important couples, since naturally their names are often mentioned close to each other --- Kitty and Levin, Anna and Vronski.
\item The weighted-window frequency finds $ \{ \text{draw}, \text{room} \} $ and $ \text{draw} \to \text{room} $ at positions 10 and 7, respecitvely. The fixed-window confidence ranked $ \{ \text{draw}, \text{room} \} $ at position 30, and the disjoint-window frequency ranked it at position 90. The weighted-window frequency ranks it highly because of the noun \emph{drawing room} (those two words always directly following each other), making for many minimal windows of great weight.
\end{itemize}

\subsection{Analysis of association rules mined from \emph{tolstoy} dataset}

% The algorithm always uses the associated frequency and confidence measures together. For instance, association rules mined by the disjoint-window frequency will generate association rules according to the minimal-window confidence.

Table~\ref{table:tolstoy-rules-par-fwi} shows the top 15 association rules consisting of parallel episodes using the fixed-window confidence.

For the fixed-window confidence, there are a lot of rules of confidence 1. In fact, in this experiment, there were $ 129 $ association rules consisting of serial episodes that had a confidence of 1, and $ 114\,230 $ such rules for parallel episodes. For minimal windows, the situation wasn't much better: $ 2\,218 $ association rules had a confidence of 1. This poses a problem for top-k techniques. The top-5 lists that we show here are just a small sample of equally-rated association rules.

In the top 5 for the fixed-window confidence, all of the rules have the same tail:
\begin{align*}
\{ \text{stepan},\allowbreak\text{arkadyevitch},\allowbreak\text{ey},\allowbreak\text{love},\allowbreak\text{youth},\allowbreak\text{gallant},\allowbreak\text{steed},\allowbreak\text{token},\allowbreak\text{declaim} \}
\end{align*}
In fact, the output contains 63 of these rules, all with a confidence of 1. This episode originates from a quote that is mentioned twice in the book (non-stemmed quotation):
\begin{quotation}
``I know a gallant steed by tokens sure,

And by his eyes I know a youth in love,''

declaimed Stepan Arkadyevitch.
\end{quotation}
The heads of these association rules all include the two event types that are outermost in the occurrence, \emph{gallant} and \emph{arkadyevitch}, and leave out different words from the middle of the quote: \emph{stepan}, \emph{ey}, \emph{love}, et cetera. All of those subepisosdes are covered by the same fixed windows as the tail, resulting in a fixed-window confidence of 1. Needless to say, including all of these rules is not very informative. Here, restricting the output to closed association rules would be very helpful. The association rule
\begin{align*}
& \{ \text{arkadyevitch},\allowbreak\text{gallant},\allowbreak\text{declaim} \} \Rightarrow \\
& \{ \text{stepan},\allowbreak\text{arkadyevitch},\allowbreak\text{ey},\allowbreak\text{love},\allowbreak\text{youth},\allowbreak\text{gallant},\allowbreak\text{steed},\allowbreak\text{token},\allowbreak\text{declaim} \}
\end{align*}
has a minimal head in order to have a confidence of 1, and the heads of all other 62 rules are superepisodes of this minimal head. So the other 62 rules provide no additional information.

\begin{table}
\begin{tabulary}{\textwidth}{R|L}
\# & association rule (fixed-window confidence) \\
\hline
1 & $ \{ \text{arkadyevitch},\allowbreak\text{ey},\allowbreak\text{love},\allowbreak\text{youth},\allowbreak\text{gallant},\allowbreak\text{steed},\allowbreak\text{token},\allowbreak\text{declaim} \} \Rightarrow \{ \text{stepan},\allowbreak\text{arkadyevitch},\allowbreak\text{ey},\allowbreak\text{love},\allowbreak\text{youth},\allowbreak\text{gallant},\allowbreak\text{steed},\allowbreak\text{token},\allowbreak\text{declaim} \} $ (1) \\
2 & $ \{ \text{stepan},\allowbreak\text{arkadyevitch},\allowbreak\text{love},\allowbreak\text{youth},\allowbreak\text{gallant},\allowbreak\text{steed},\allowbreak\text{token},\allowbreak\text{declaim} \} \Rightarrow \{ \text{stepan},\allowbreak\text{arkadyevitch},\allowbreak\text{ey},\allowbreak\text{love},\allowbreak\text{youth},\allowbreak\text{gallant},\allowbreak\text{steed},\allowbreak\text{token},\allowbreak\text{declaim} \} $ (1) \\
3 & $ \{ \text{arkadyevitch},\allowbreak\text{love},\allowbreak\text{youth},\allowbreak\text{gallant},\allowbreak\text{steed},\allowbreak\text{token},\allowbreak\text{declaim} \} \Rightarrow \{ \text{stepan},\allowbreak\text{arkadyevitch},\allowbreak\text{ey},\allowbreak\text{love},\allowbreak\text{youth},\allowbreak\text{gallant},\allowbreak\text{steed},\allowbreak\text{token},\allowbreak\text{declaim} \} $ (1) \\
4 & $ \{ \text{arkadyevitch},\allowbreak\text{ey},\allowbreak\text{youth},\allowbreak\text{gallant},\allowbreak\text{steed},\allowbreak\text{token},\allowbreak\text{declaim} \} \Rightarrow \{ \text{stepan},\allowbreak\text{arkadyevitch},\allowbreak\text{ey},\allowbreak\text{love},\allowbreak\text{youth},\allowbreak\text{gallant},\allowbreak\text{steed},\allowbreak\text{token},\allowbreak\text{declaim} \} $ (1) \\
5 & $ \{ \text{stepan},\allowbreak\text{arkadyevitch},\allowbreak\text{youth},\allowbreak\text{gallant},\allowbreak\text{steed},\allowbreak\text{token},\allowbreak\text{declaim} \} \Rightarrow \{ \text{stepan},\allowbreak\text{arkadyevitch},\allowbreak\text{ey},\allowbreak\text{love},\allowbreak\text{youth},\allowbreak\text{gallant},\allowbreak\text{steed},\allowbreak\text{token},\allowbreak\text{declaim} \} $ (1) \\
\end{tabulary}
\caption{Top 5 parallel association rules, by the fixed-window confidence.}
\label{table:tolstoy-rules-par-fwi}
\end{table}

Table~\ref{table:tolstoy-rules-par-wwi} shows the top 5 association rules consisting of parallel episodes using the weighted-window confidence, and Table~\ref{table:tolstoy-rules-ser-wwi} shows the same for serial episodes.

From this top 5 we mostly learn that for certain characters, the full names are always used.

\begin{itemize}
\item Parallel \#1: Whenever two occurrences of \emph{arkadyevitch} are close together, \emph{stepan} can be found nearby. This is because the character \emph{Stepan Arkadyevitch} is never addressed by his first name only.
\item Parallel and serial \#2: The same for \emph{Alexei Alexandrovitch}.
\item Parallel \#3 and \#5 give more or less the same information as \#1.
\item Serial \#1: $ \text{sergei} \to \text{levin} $ leads consistenly to $ \text{sergei} \to \text{ivanovitch} \to \text{levin} $. Sergei Ivanovitch Levin is one character's full name, so it makes sense that whenever we find the outermost parts, the inner part is in between.
\item ...
\end{itemize}

For the weighted-window confidence, there were far fewer association rules in the top 15 with the same confidence value.

However, we should note that the comparison is not entirely fair --- the sets of episodes upon which we built the association rules depended on the time constraints of the respective frequency measures. Mining parallel episodes based on their fixed-window frequency

If we wanted to make a better qualitative comparison between the different confidence measures, perhaps we should decide on a set of episodes, and generate confidence rules based on those. While not hard to realize, this would require some time to modify the implementation, and could be subject of future work.

Because of the aforementioned, it is hard to make definitive statements about how the confidence measures compare in quality. We did, however, see that some elimination of redundancy would undoubtedly benefit all three confidence measures.

\begin{table}
\begin{tabulary}{\textwidth}{R|L}

\# & association rule (weighted-window confidence) \\
\hline
1 & $ \{ \text{arkadyevitch}, \text{arkadyevitch} \} \Rightarrow \{ \text{stepan}, \text{arkadyevitch}, \text{arkadyevitch} \} $ (1) \\
2 & $ \{ \text{alexandrovitch}, \text{alexandrovitch} \} \Rightarrow \{ \text{alexei}, \text{alexandrovitch}, \text{alexandrovitch} \} $ (1) \\
3 & $ \{ \text{stepan}, \text{stepan} \} \Rightarrow \{ \text{stepan}, \text{stepan}, \text{arkadyevitch} \} $ (1) \\
4 & $ \{ \text{countess}, \text{ivanovna} \} \Rightarrow \{ \text{countess}, \text{lidia}, \text{ivanovna} \} $ (0.943) \\
5 & $ \{ \text{arkadyevitch}, \text{alexei} \} \Rightarrow \{ \text{stepan}, \text{arkadyevitch}, \text{alexei} \} $ (0.937) \\

\end{tabulary}
\caption{Top 5 parallel association rules, by the weighted-window confidence.}
\label{table:tolstoy-rules-par-wwi}
\end{table}
\begin{table}
\begin{tabulary}{\textwidth}{R|L}

\# & weighted-window confidence \\
\hline
1 & $ \text{sergei} \to \text{levin} \Rightarrow \text{sergei} \to \text{ivanovitch} \to \text{levin} $ (1) \\
2 & $ \text{alexandrovitch} \to \text{alexandrovitch} \Rightarrow \text{alexandrovitch} \to \text{alexei} \to \text{alexandrovitch} $ (1) \\
3 & $ \text{stepan} \to \text{levin} \Rightarrow \text{stepan} \to \text{arkadyevitch} \to \text{levin} $ (1) \\
4 & $ \text{stepan} \to \text{smile} \Rightarrow \text{stepan} \to \text{arkadyevitch} \to \text{smile} $ (1) \\
5 & $ \text{alexei} \to \text{alexandrovitch} \to \text{alexandrovitch} \Rightarrow \text{alexei} \to \text{alexandrovitch} \to \text{alexei} \to \text{alexandrovitch} $ (1) \\

\end{tabulary}
\caption{Top 5 serial association rules, by the weighted-window confidence.}
\label{table:tolstoy-rules-ser-wwi}
\end{table}

\subsection{Comparison with non-frequency-based methods}

\iffalse
As we saw in Section~\ref{sec:experiments-quality-episodes}, anti-monotonic frequency measures inherently put larger episodes at a disadvantage, since an episode is never more frequent than any of its subepisodes.
\fi

So far, in this thesis, we have only studied frequency-based interestingness measures. In this section, we'll look into a different interpretation of interestingness: \emph{cohesion}. The cohesiveness of an episode expresses how closely the events of its occurrences are to each other. An episode is deemed cohesive in regard to a sequence if the events that constitute occurrences are close to each other; if, in other words, its minimal windows are generally small.

All frequency measures we have studied so far had a cohesion angle as well. This is natural in the context of a long event sequence, where events that are far apart, are unlikely to have anything to do with each other. The width of the sliding window was not only a means to make the algorithms computationally feasible, it also placed a cutoff beyond which events were deemed uncorrelated. For the disjoint-window frequency, this cutoff was quite harsh --- all minimal windows within a certain width $ \rho $ were considered equally valuable, while any windows larger than $ \rho $ were judged uncorrelated.

The weighted-window frequency takes into account the width of the minimal windows more carefully --- it assigns a weight to each minimal window (the inverse of its width) and sums the weights. So, the weighted-window frequency places importance on the cohesiveness of patterns, but it remains a measure based on frequency.

With cohesion-based interestingness measures, the main focus shifts towards the cohesion of patterns.

In~\cite{cule2016efficient}, the cohesion of an episode $ \alpha $ is defined as
\begin{align*}
C(\alpha) = \frac{| \alpha |}{\overline{W}(\alpha)}
\end{align*}
where $ \overline{W}(\alpha) $ is the average width of the minimal windows of $ \alpha $. Further details on the exact definition can be found in~\cite{cule2016efficient}.

This definition makes an effort not to disadvantage larger patterns --- as anti-monotonic frequency measures inherently do for computational reasons --- by defining the cohesiveness of an episode to be proportional in the size of the episode. The paper describes a method for mining episodes according to this non-monotonic measure, using other techniques of pruning the search space, which are beyond the scope of this thesis.

While cohesiveness is not frequency-based, there is a frequency aspect, still: in the mining method in~\cite{cule2016efficient}, the event types that make up a cohesive episode should be frequent by themselves, given a support threshold. Event types that don't have enough support won't appear in any pattern. This has a computational advantage, because it effectively reduces the size of the alphabet. It also prevents including patterns that are too infrequent to bear any kind of significance.

The mining algorithm from~\cite{cule2016efficient}, called \textsc{Fci}, mines parallel episodes with an injective $ lab $-function, meaning that each event type appears at most once. The algorithm takes the following parameters:
\begin{itemize}
\item The minimal support that an event type must have in order to be considered at all.
\item The maximal size of patterns to generate.
\item The minimal cohesion that any pattern must have.
\end{itemize}

In~\cite{cule2016efficient}, the cohesion of a pattern is defined using the mean width of the minimal windows. The mean has a number of downsides, though; one of which is that the mean of a distribution is unstable if there are outliers. So for instance, if one window is significantly larger or smaller than the others, the cohesion may be greatly affected. Therefore, a quantile-based approach was proposed in~\citep{feremans2018mining}. Here, the minimal occurrences are grouped by width using a threshold. The quantile-based cohesion is then defined as the percentage of minimal occurrences that are smaller than the threshold. The threshold is proportional to the size of the episode, again in an effort not to disadvantage larger patterns.

Their mining method \textsc{Qcsp} mines serial episodes, and takes the following parameters:
\begin{itemize}
\item The minimal support that an event type must have in order to be considered at all.
\item The maximal size of patterns to generate.
\item A minimal-window width threshold (as mentioned above, which gets multiplied by the size of each episode being considered).
\item $ k $. The algorithm reports only the $ k $ patterns with the greatest cohesion.
\end{itemize}

We ran \textsc{Fci} with a minimum support of 5, a maximal pattern size of 5, and a minimal cohesion of 0.014 (a lower threshold started to take significantly more time); and \textsc{Qcsp} with a minimum support of 5, a maximal pattern size of 7 and a window-width threshold of 2.

% TODO check window-width threshold

\begin{table}
\centering

\begin{tabulary}{\textwidth}{R|L|L}

\# & cohesion & quantile-based cohesion \\
\hline
1 & $ \{ \text{agafea}, \text{mihalovna} \} $ (1) & $ \text{stepan} \to \text{arkadyevitch} $ (0.998, 1096) \\
2 & $ \{ \text{char}, \text{banc} \} $ (1) & $ \text{alexei} \to \text{alexandrovitch} $ (0.95, 1203) \\
3 & $ \{ \text{pinc}, \text{nez} \} $ (1) & $ \text{sergei} \to \text{ivanovitch} $ (0.965, 603) \\
4 & $ \{ \text{bell}, \text{soeur} \} $ (1) & $ \text{agafea} \to \text{mihalovna} $ (1, 148) \\
5 & $ \{ \text{stepan}, \text{arkadyevitch} \} $ (0.915) & $ \text{darya} \to \text{alexandrovna} $ (0.967, 424) \\
6 & $ \{ \text{nativ}, \text{tribe} \} $ (0.523) & $ \text{lidia} \to \text{ivanovna} $ (0.977, 221) \\
7 & $ \{ \text{lizaveta}, \text{petrovna} \} $ (0.408) & $ \text{lizaveta} \to \text{petrovna} $ (0.98, 49) \\
8 & $ \{ \text{ivanovna}, \text{lidia} \} $ (0.137) & $ \text{nativ} \to \text{tribe} $ (0.966, 29) \\
9 & $ \{ \text{alexandrovitch}, \text{alexei} \} $ (0.05) & $ \text{liza} \to \text{merkalova} $ (0.706, 34) \\
10 & $ \{ \text{sergei}, \text{ivanovitch} \} $ (0.0428) & $ \text{marya} \to \text{nikolaevna} $ (0.673, 98) \\
11 & $ \{ \text{darya}, \text{alexandrovna} \} $ (0.0363) & $ \text{countess} \to \text{lidia} \to \text{ivanovna} $ (0.586, 389) \\
12 & $ \{ \text{bezzubov}, \text{landau} \} $ (0.0351) & $ \text{vassili} \to \text{lukitch} $ (0.588, 51) \\
13 & $ \{ \text{gladiat}, \text{frou} \} $ (0.0326) & $ \text{countess} \to \text{lidia} $ (0.557, 280) \\
14 & $ \{ \text{partnership}, \text{ryezunov} \} $ (0.0303) & $ \text{countess} \to \text{ivanovna} $ (0.549, 277) \\
15 & $ \{ \text{bridal}, \text{lectern} \} $ (0.0281) & $ \text{madam} \to \text{stahl} $ (0.55, 171) \\

\end{tabulary}

\caption{The top 15 patterns mined from~\emph{tolstoy} using cohesion (\textsc{Fci}, minimum support 5, maximal size 5) and quantile-based cohesion (\textsc{Qcsp}, minimum support 7, maximal size 5).}
\label{table:cohesive-patterns}
\end{table}

Table~\ref{table:cohesive-patterns} shows the 15 most interesting episodes by the two different definitions of cohesion. The implementation of the quantile-based cohesion sorts episodes according to a linear combination of the cohesion and the support.

Some of these patterns are both frequent and cohesive; as we already encountered them in the output of our implementation. We see some of the characters' names reappear. But we also see many patterns that we haven't seen before.

Interestingly, for the regular cohesion, there are some \emph{very} cohesive patterns consisting of words that don't appear on their own anywhere else --- such as \emph{belle-soeur}, which appears only five times in the book, just satisfying our support threshold --- after which the cohesion drops sharply.

[TODO highlight some more patterns, draw conclusion]
