\chapter{Nederlandstalige samenvatting}

Het doel van datamining is doorgaans om nuttige informatie uit grote datasets te halen -- datasets die te groot zijn om manueel in detail door te kammen. In dit werk spitsen we ons toe op sequential pattern mining, waarbij we patronen zoeken in sequentiële gegevens, meer specifiek: gegevens die voor te stellen zijn als een \emph{event sequence}.

Na het uiteenzetten van een formeel kader met verschillende soorten patronen en manieren om te kwantificeren hoe interessant zo'n patroon is, beschrijven we een algoritme dat, gegeven een event sequence, alle patronen vindt die volgens een gegeven drempelwaarde interessant genoeg zijn.

Vervolgens testen we de efficiëntie van eem implementatie van het algoritme onder verschillende event sequences en parameters, en bestuderen we of we daarmee werkelijk interessante patronen vinden.
